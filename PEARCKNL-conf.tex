\documentclass[sigconf]{acmart}

\usepackage{booktabs} % For formal tables


% Copyright
%\setcopyright{none}
%\setcopyright{acmcopyright}
%\setcopyright{acmlicensed}
\setcopyright{rightsretained}
%\setcopyright{usgov}
%\setcopyright{usgovmixed}
%\setcopyright{cagov}
%\setcopyright{cagovmixed}


% DOI
\acmDOI{10.475/123_4}

% ISBN
\acmISBN{123-4567-24-567/08/06}

%Conference
% JRM - updated by editor or typesetter
\acmConference[shortname]{name}{date}{venue}
% JRM - updated by editor or typesetter
\acmYear{ACMYr}
% JRM - updated by editor or typesetter
\copyrightyear{CRYr}

% JRM - updated by editor or typesetter
\acmPrice{15.00}


\begin{document}
\title{Performance Benchmarking of the R Programming Environment on the Stampede 2 Supercomputer}
%\titlenote{Produces the permission block, and
%  copyright information}
%\subtitle{Extended Abstract}
%\subtitlenote{The full version of the author's guide is available as
%  \texttt{acmart.pdf} document}


\author{James R. McCombs}
%\authornote{}
%\orcid{}
\affiliation{%
  \institution{Indiana University}
  \department{Pervasive Technology Institute}
  \streetaddress{2709 E 10th St.}
  \city{Bloomington}
  \state{Indiana}
  \postcode{47405}
}
\email{jmccombs@iu.edu}

\author{Scott Michael}
%\authornote{}
%\orcid{}
\affiliation{%
  \institution{Indiana University}
  \department{Pervasive Technology Institute}
  \streetaddress{2709 E 10th St.}
  \city{Bloomington}
  \state{Indiana}
  \postcode{47405}
}
\email{scamicha@iu.edu}

% The default list of authors is too long for headers}
\renewcommand{\shortauthors}{J.R. McCombs et al.}


\begin{abstract}
This paper provides a sample of a \LaTeX\ document which conforms,
somewhat loosely, to the formatting guidelines for
ACM SIG Proceedings.\footnote{This is an abstract footnote}
\end{abstract}

%
% The code below should be generated by the tool at
% http://dl.acm.org/ccs.cfm
% Please copy and paste the code instead of the example below.
%
\begin{CCSXML}
<ccs2012>
<concept>
<concept_id>10010405.10010432</concept_id>
<concept_desc>Applied computing~Physical sciences and engineering</concept_desc>
<concept_significance>300</concept_significance>
</concept>
<concept>
<concept_id>10010405.10010432.10010442</concept_id>
<concept_desc>Applied computing~Mathematics and statistics</concept_desc>
<concept_significance>100</concept_significance>
</concept>
<concept>
<concept_id>10010520.10010521.10010528.10010536</concept_id>
<concept_desc>Computer systems organization~Multicore architectures</concept_desc>
<concept_significance>500</concept_significance>
</concept>
</ccs2012>
\end{CCSXML}

\ccsdesc[300]{Applied computing~Physical sciences and engineering}
\ccsdesc[100]{Applied computing~Mathematics and statistics}
\ccsdesc[500]{Computer systems organization~Multicore architectures}

% We no longer use \terms command
%\terms{Theory}

\keywords{benchmarking, many-core, R, scalability, Stampede supercomputer, Xeon Phi, XSEDE}


\maketitle

\section{Introduction}
% TODO: Move sections 1 and 2 of white paper into introduction


\section{Methodology and Approach}
% TODO: Methodology/approach section after introduction
  % Pull in from white paper section 3:
  %   1. focus on dense matrix, not sparse
  %   2. Broad discussion of matrix sizes, thread counts, strong scaling
  %   3. emphasize what we were getting at: SNB vs. KNL performance
  %      for similar thread counts and scalability

\section{Results}
% Part c of white paper section 4 to be moved into results section
% Move white paper section sec. into results
%  Keep matrix dimension and experiment parameter specifics
%  in the results section (don't want this detail in Methodology and Approach)
We conducted single-node strong scaling performance tests for each benchmark
  function over a wide range of problem dimensions on each compute node
  configuration.
We configured multithreading of each test through a set of default global
  configuration parameters that applied to all tests, and a separate set of
  parameters specific to each test.
The default global parameters, which were implemented as environment variables,
  are listed in Table~\ref{tab:threading}.
The chosen thread affinity, controlled by \texttt{KMP\_AFFINITY} is recommended
  in the MKL developer’s guide~\cite{intel:mkl2017}.
We disabled the dynamic thread adjustment controlled by \texttt{OMP\_DYNAMIC}
  and \texttt{KMP\_DYNAMIC} so that the scaling tests could be conducted
  ~\cite{intel:cpp2015, intel:cpp2017, intel:mkl11_2, intel:mkl2017}.
We tested the scalability by increasing the number of OpenMP threads and MKL
  threads through the test-specific environment variables 
  \texttt{OMP\_NUM\_THREADS} and \texttt{MKL\_NUM\_THREADS}, respectively.
The performance tests on the Sandy Bridge nodes included the two Xeon Sandy
  Bridge processors, but did not include any co-processing by the Xeon Phi
  coprocessor. 

\begin{table*}
  \caption{Default Global Multithreading Parameters}
  \label{tab:threading}
  \begin{tabular}{lll}
    \toprule
    Multithreading Parameter & Description & Setting\\
    \midrule
    \texttt{KMP\_AFFINITY}   & Controls binding of OpenMP threads to hardware resources & \texttt{granularity=fine,compact,1,0}\\
    \texttt{OMP\_DYNAMIC}    & Enables (\texttt{true}) or disables (\texttt{false}) the dynamic adjustment   & \texttt{false}\\
                             & of the number of threads created by OpenMP &\\
    \texttt{KMP\_DYNAMIC}    & Enables (\texttt{true}) or disables (\texttt{false}) the dynamic adjustment & \texttt{false} \\
                             & of the number of threads created by the MKL &\\
    \bottomrule
  \end{tabular}
\end{table*}


Small matrix dimensions of size $2000$ and $4000$ were often sufficient to
  determine performance cross over points for the kernel functions where one
  node type would begin to outperform the other given an increase in matrix
  size.
Larger matrices were useful for determining which kernels efficiently utilized a
  large number of threads on the Knights Landing nodes.
Hyper-threading was not enabled on the Sandy Bridge processors, but it was
  enabled on the Knights Landing nodes.
We observed no improvements in scalability by placing more than one thread on
  each core of the Knights Landing processors, but have included results for
  more than 68 threads for completeness.
The numbers of threads we tested on the Sandy Bridge nodes were:
  $1$, $2$, $4$, $8$, $12$, and $16$; and the numbers of threads we tested on
  the Knights Landing nodes were:
  $1$, $2$, $4$, $8$, $16$, $34$, $66$, $68$, $136$, $204$, and $272$.
Apart from some exceptions, we tested the matrix algebra kernels with square
  matrices having the following row dimensions $N$:
  $1000$, $2000$, $4000$, $8000$, $10000$, $15000$, $20000$, and $40000$; in the
  case of the least squares fit kernel, the matrix dimensions were
  $2N \times N/2$ for the same values of $N$.
For each combination of matrix dimension and number of threads for a given
  microbenchmark, we collected the run time and computed the strong scaling as
  the ratio of the run time of the microbenchmark to the run time of the
  microbenchmark with one thread.
We attempted to run performance tests up to matrix dimensions of $30000$ and
  $40000$ on the Sandy Bridge and Knights Landing nodes, respectively; however,
  some kernels could not be tested with these dimensions due to the memory
  capacity and internal size limitations imposed by R.
The test results for each microbenchmark, matrix size, and number of threads are
  given in a spreadsheet accompanying this document.
We included plots showing microbenchmark run-time and strong scaling performance
  on Sandy Bridge and Knights Landing in the appendix; we plotted results for
  matrix dimensions that reveal when the two node types become competitive for
  a microbenchmark, or to highlight when a node type consistently outperforms
  its competitor.
In the plot legend entries, \textit{S1} indicates Sandy Bridge node performance
  and \textit{S2} indicates Knights Landing node performance; entries postfixed
  with a 15 or a 17 indicate tests were performed with the Intel 15 Update 2 or
  Intel 17 Update 1 compiler and MKL bundles, respectively.
In addition to the computational kernels, we performance tested clustering and
  neural network training functionality from the \textit{nnet} and
  \textit{cluster} packages.

After we executed a given microbenchmark for a matrix dimension and all numbers
  of threads to be tested on a given node type, we determined the minimum run
  time over all tested numbers of threads.
We computed the minimum run times in this manner for each combination of node
  type, microbenchmark, and matrix dimension.
From these minimum values, we determined the fastest node type for each
  microbenchmark and matrix dimension so that we could quantify relative
  performance as the ratio of the minimum run time achieved by the slowest node
  to that achieved by the fastest node.
In the microbenchmark performance assessments that follow, the assessments are
  grouped by microbenchmarks that exhibited similar performance characteristics,
  and we provide relative performance ratios over the tested matrix dimensions.
We also provide the percentage of the maximum possible speedup, which we define
  as the speedup divided by the number of cores, achieved by each microbenchmark
  with the largest matrix tested.
Unless stated otherwise, the Sandy Bridge performance assessments are based on
  microbenchmarks built under the Intel 15 Update 2 bundle, and the Knights
  Landing performance assessments are based on microbenchmarks built under the
  Intel 17 Update 1 bundle. 

The Sandy Bridge nodes consistently outperformed the Knights Landing nodes when
  running the Cholesky factorization (Figure 1), dense linear solve (Figure 5),
  and matrix determinant (Figure 7) kernels for the small matrix dimensions
  tested and with sixteen or fewer threads.
Comparing the best run times achieved by each node type, the Sandy Bridge nodes
  executed the Cholesky factorization up to $2.00$ times faster for tested
  matrix dimensions of $2000$ or less, the dense linear solve $1.77$ times
  faster for matrix dimensions of $1000$, and the matrix determinant $1.80$
  times faster for matrix dimensions of $1000$.
The Knights Landing nodes began to outperform the Sandy Bridge nodes for matrix
  dimensions at or above $2000$ or $4000$ for the same kernels, achieving run
  times that were $4.03$ (Cholesky factorization), $4.41$ (dense linear solve),
  and $5.23$ (matrix determinant) times faster than those achieved on the Sandy
  Bridge nodes given matrix dimensions of $30000$, $20000$, and $40000$,
  respectively.
The Sandy Bridge nodes achieved at most $65\%$, $81\%$, and $84\%$ of the
  maximum possible speedup with sixteen or fewer threads executing the Cholesky
  factorization kernel with tested matrix dimensions of $30000$ or less, the
  dense linear solve kernel with tested matrix dimensions of $20000$ or less,
  and the matrix determinant kernel with tested matrix dimensions of $40000$ or
  less, respectively.
The Knights Landing nodes achieved at most $38\%$, $48\%$, and $65\%$ of the
  maximum possible speedup for the same kernels and matrix dimensions.

The eigendecomposition kernel (Figure 2) performed up to $2.75$ times faster on
  the Sandy Bridge nodes than on the Knights Landing nodes for tested matrix
  dimensions of $8000$ or less.
As we increased the matrix dimension beyond $8000$, the Knights Landing nodes
  were up to $1.54$ times faster.
It is also interesting to note that the eigendecomposition kernel never scaled
  beyond eight threads on the Sandy Bridge nodes, regardless of matrix
  dimension, but the performance did not deteriorate as the maximum number of
  threads (sixteen) was reached either.
For tested matrix dimensions of $20000$ or less, the Knights Landing nodes
  achieved at most $5\%$ of the maximum possible speedup; the Sandy Bridge
  nodes achieved at most $17\%$ of the maximum possible speedup for the same
  matrix dimensions.

The least squares fit kernel (Figure 3) experienced difficulties scaling beyond
  four threads on the Sandy Bridge nodes.
Recall that the matrix dimensions in the case of the least squares fit are
  parameterized by $N$ such that the matrix is $2N \times N/2$.
The Sandy Bridge run times were up to $2.18$ times faster than those of the
  Knights Landing nodes for tested values of $N=8000$ or less, and the Sandy
  Bridge nodes achieved at most $10\%$ of the maximum possible speedup for
  tested values of $N=20000$ or less.
The Knights Landing run times were up to $1.46$ times faster than the Sandy
  Bridge nodes for tested matrix dimensions greater than $8000$.
The strong scaling on Knights Landing stagnated beyond eight threads, achieving
  only $5\%$ of the maximum possible speedup for tested values of $N=20000$ or
  less.
Based on this analysis and the stagnant or degraded strong scaling shown in
  Figure 3, we concluded that the linear least squares fit kernel is not
  scalable beyond a few threads on either node type.

The Knights Landing nodes were consistently competitive with or outperformed the
  Sandy Bridge nodes for all matrix dimensions and numbers of threads tested
  with the matrix cross product (Figure 6) and matrix-matrix multiplication
  (Figure 8) kernels.
The matrix cross product kernel was up to $4.27$ times as fast on the Knights
  Landing nodes than on the Sandy Bridge nodes for tested matrix dimensions of
  $40000$ or less.
The matrix-matrix multiplication kernel was up to $3.12$ times as fast on the
  Knights Landing nodes than on the Sandy Bridge nodes for tested matrix
  dimensions greater than $2000$; for tested matrix dimension of $2000$ or less,
  the best Sandy Bridge nodes were up to $2.11$ times faster than the Knights
  Landing nodes.
The Knights Landing and Sandy Bridge nodes executing the matrix cross product
  kernel achieved $51\%$ and $84\%$ of their maximum possible speedups,
  respectively.
The Knights Landing and Sandy Bridge nodes executing the matrix-matrix
  multiplication kernel achieved $43\%$ and $96\%$ of their maximum possible
  speedups, respectively. 

The matrix-vector multiplication kernel on the Sandy Bridge nodes consistently
  outperformed the Knights Landing nodes for all tested matrix dimensions and
  numbers of threads (Figure 9).
The Sandy Bridge nodes ran the kernel up to $6.14$ times faster than the
  Knights Landing nodes for matrix dimensions of $40000$ or less.
The speedups achieved by the Knights Landing and Sandy Bridge nodes with the
  kernel were at most $2\%$ and $9\%$, respectively.
The poor scalability may be because the single-core performance is very high,
  and matrices of larger dimension could not be tested to increase the workload
  per core because of the matrix dimension limitations imposed by R.

We tested the QR decomposition kernel using both the LINPACK and LAPACK
  routines.
The QR kernel executed with the LAPACK implementation performed the best on both
  nodes, the results for which are given in Figure 10.
The kernel performed up to $1.48$ times faster on the Sandy Bridge nodes for
  matrix dimensions of $2000$ or less.
For matrix dimensions of $4000$ or more, the Knights Landing nodes were up to
  $6.32$ times faster than the Sandy Bridge nodes.
The Knights Landing nodes achieved speedups of at most $28\%$ of the maximum
  possible, and the Sandy Bridge nodes achieved speedups of at most $38\%$ of
  the maximum possible. 

The singular value decomposition kernel exhibited similar performance
  (Figure 11).
For each tested matrix dimension of $2000$ or less, the Sandy Bridge nodes were
  up to $1.78$ times faster than the Knights Landing nodes.
For tested matrix dimensions of $4000$ or greater, the Knights Landing nodes
  were up to $4.25$ times as fast as those achieved on the Sandy Bridge nodes.
The percentages of the maximum possible speedup achieved by the Knights Landing
  and Sandy Bridge nodes were at most $23\%$ and $22\%$, respectively.  

We also tested the performance of the Cholesky factorization, dense linear
  solve, and matrix cross product kernels when run exclusively out of the
  high-bandwidth MCDRAM.
We accomplished this by executing the kernels on Knights Landing nodes booted in
  flat mode and used the numactl utility with the \texttt{--preferred} option
  set to indicate that as much memory as possible for the R programming
  environment was to be allocated in the MCDRAM NUMA node; the environment and
  the dynamically allocated matrices we tested did fit entirely in MCDRAM.
The performance results, plotted in Figure 12, show no appreciable differences
  in kernel performance when the MCDRAM is configured in cache mode or flat
  mode.

We conducted additional experiments with the Sandy Bridge nodes to determine if
  the least squares fit, dense linear solve, or QR decomposition microbenchmarks
  would show improvement under the Intel 17 Update 1 compiler and MKL.
The dense linear solve and QR decomposition microbenchmark results showed little
  or no improvement over their Intel 15 Update 2 counterparts; however, the
  least squares fit kernel exhibited improved scalability beyond four threads
  for moderate to large matrix dimensions (Figure 4).   

We performance tested the \texttt{nnet} function from the \textit{nnet} package
  and the \texttt{pam} function of the \textit{cluster} package to determine if
  those commonly used capabilities were parallelized.
We chose the neural network training parameters listed in Table~\ref{tab:nnetParams} to
  performance test the \texttt{nnet} function because they required the
  underlying solver to perform several iterations to achieve the required
  convergence.
Our tests revealed that the neither of the package functions parallelized, as
  was indicated by the run times and analysis tools used to determine if
  multiple threads were being created to perform the computation.
Nevertheless, the tests are useful for assessing performance of commonly used R
  libraries on each node type.
The Sandy Bridge nodes were approximately $3.5$ times faster and $3.0$ times
  faster than the Knights Landing nodes at training the neural net with three
  features and five features (Table~\ref{tab:nnetResults}), respectively.
Note that the run times decreased when we increased the number of feature
  vectors from $5000$ to $10000$ and the feature vector was of dimension three.
This is because neural net training solves a non-convex objective function that
  can vary greatly with the number of training vectors and features; therefore,
  it is useful only to compare performance between the two node types for a pair
  of features and number of training vectors.

We used the package default parameters to test the \texttt{pam} clustering
  benchmark function to compute the membership of feature vectors to seven
  different clusters as described in Section III.
The Sandy Bridge nodes were approximately $1.34$-$1.44$ times faster than the
  Knights Landing nodes when running the pam function for up to $17500$ feature
  vectors, but the Knights Landing nodes closed that gap to approximately $1.04$
  times as fast for $35000$ feature vectors.
With $40005$ feature vectors, the Knights Landing nodes were $1.11$ times faster
  than the Sandy Bridge nodes.

% nnet training parameters
\begin{table*}
  \caption{Training Parameters Used with Neural Network Training Benchmark}
  \label{tab:nnetParams}
  \begin{tabular}{lll}
    \toprule
    Training parameters & Description & Value\\
    \midrule
      \texttt{size}     & Number of neurons in hidden layer        & $200$ \\
      \texttt{maxit}    & Maximum number of solver iterations      & $12500$ \\
      \texttt{abstol}   & Absolute tolerance for the fit criterion & $1.0\times 10^{-6}$\\
      \texttt{reltol}   & Tolerance used to terminate solver if    & $1.0\times 10^{-11}$\\
                        & the fit criterion does not decrease by a &\\
                        & factor of at least $(1-\mathrm{\texttt{reltol}})$ &\\
      \texttt{linout}   & Enables real-valued, linear outputs      & \texttt{TRUE}\\
      \texttt{MaxNWts}  & The maximum allowable number of weights  & $4000$\\ 
    \bottomrule
  \end{tabular}
\end{table*}

% Neural networking benchmark results
\begin{table*}
  \caption{Run Time of Neural Network Training Benchmark With a Single Thread}
  \label{tab:nnetResults}
  \begin{tabular}{llll}
    \toprule
      Number of features & Number of training vectors & \multicolumn{2}{c}{Run time (sec)}\\
                         &                            & Sandy Bridge & Knights Landing\\
    \midrule
    $3$ & $5000$  & $3.239\times 10^{2}$ & $1.156\times 10^{3}$ \\
    $3$ & $10000$ & $1.842\times 10^{2}$ & $6.690\times 10^{2}$ \\
    $5$ & $5000$  & $3.520\times 10^{2}$ & $1.028\times 10^{3}$ \\
    $5$ & $10000$ & $1.536\times 10^{3}$ & $4.664\times 10^{3}$ \\
    $5$ & $15000$ & $2.660\times 10^{3}$ & $8.216\times 10^{3}$ \\
    \bottomrule
  \end{tabular}
\end{table*}

% Clustering benchmark results
\begin{table*}
  \caption{Run Time of Clustering Benchmark for a Single Thread}
  \label{tab:clusterResults}
  \begin{tabular}{lllll}
    \toprule
    Number of features & Number of clusters & Number of feature vectors & \multicolumn{2}{c}{Run time (sec)}\\
                       &                    &                           & Sandy Bridge & Knights Landing\\
    \midrule
    $3$ & $7$ & $7000$  & $4.416\times 10^{1}$ & $5.949\times 10^{1}$ \\
    $3$ & $7$ & $10500$ & $1.065\times 10^{2}$ & $1.536\times 10^{2}$ \\
    $3$ & $7$ & $14000$ & $2.890\times 10^{2}$ & $4.112\times 10^{2}$ \\
    $3$ & $7$ & $17500$ & $6.757\times 10^{2}$ & $9.326\times 10^{2}$ \\
    $3$ & $7$ & $35000$ & $2.578\times 10^{3}$ & $2.698\times 10^{3}$ \\
    $3$ & $7$ & $40005$ & $9.236\times 10^{3}$ & $8.313\times 10^{3}$ \\
    \bottomrule
  \end{tabular}
\end{table*}

%

The overheads in the kernel functions can potentially be large compared to
  similar functionality implemented exclusively in C due to data copying and
  validity checking that is common in interpreted languages.
The benchmark kernel functions are essentially wrapper functions, most of which
  call BLAS and LAPACK routines to perform the bulk of the
  computation~\cite{cran:Rmanuals}.
To determine the potential for further optimization of R functionality, we
  developed a set of drivers written to test the performance of a subset of the
  benchmark kernel functions in C and compared their performance to that of
  their R interpreter counterparts.
We developed C drivers for the matrix cross product, QR decomposition, and
  dense linear solve with multiple right-hand sides.
These kernel functions are each implemented as R language wrapper functions
  which call C internal functions.
The cross product internal function is implemented as a call to the LAPACK
  FORTRAN function \texttt{dsyrk} followed by a for loop to copy the upper
  triangle of the result matrix to its lower triangle.
The QR decomposition internal function is implemented as a call to the LAPACK
  FORTRAN function \texttt{dgeqp3}, and the dense linear solve internal function
  is implemented as a call to the LAPACK FORTRAN function \texttt{dgesv}.
The C driver for QR calls the \texttt{LAPACKE\_dsyrk} C wrapper function which
  dynamically allocates an optimally sized workspace, the other driver functions
  call the FORTRAN functions directly.
The R implementation of the dense linear solve computes an estimate of the
  condition number by default, a task we did not implement in the corresponding
  C driver.
The performance of the C drivers is given in Figure 13, Figure 14, and
  Figure 15.
The results show that the performance of R kernels are highly competitive with
  the native C drivers for all matrix dimensions and threads.
The performance curves for the dense linear solve applied to the smallest tested
  matrix dimensions (Figure 15) do show some performance anomalies, however.
There is a substantial difference between the strong scaling of the C driver and
  the R kernel when using two threads.
Furthermore, the C driver scalability diverges sharply from that of the R kernel
  for eight or more threads for matrix dimensions $1000$ and $2000$.
We reran the strong scaling test of the C driver several times for the smaller
  matrix dimensions to verify the results, with no appreciable change in the
  curves.

\section{Discussion}
We developed a new R HPC benchmarking package consisting of microbenchmarks of
  several computational kernels and benchmarks of machine learning functions
  from the R standard distribution.
The implementation of the benchmarks improves upon limitations of the few R
  benchmarks that have been developed to date, namely the inability to perform
  scalability studies over successively larger problem sizes and to
  independently configure how each individual benchmark is executed.   
We obtained performance results running our HPC benchmark on both the Sandy
  Bridge and Knights Landing nodes of the Stampede supercomputer at the Texas
  Advanced Computing Center.
We performed the tests using the latest Intel compiler and MKL bundle available
  to users on the Sandy Bridge nodes and the Intel 17 Update 1 bundle on the
  Knights Landing nodes.
With few exceptions, the Sandy Bridge nodes achieved the best run-time
  performance for matrix dimensions up to a few thousand, an expected result
  given that the clock speed of the Sandy Bridge CPUs is roughly twice that of
  the Knights Landing CPUs.
However, the best run time achieved on the Sandy Bridge nodes was never more
  than three times as fast as the best run time achieved on the Knights Landing
  nodes for a given matrix dimension, except in the case of matrix-vector
  multiplication, for which the Sandy Bridge nodes were over six times faster
  than the Knights Landing nodes.
When we tested the Cholesky factorization, dense linear solve, matrix
  determinant, matrix-matrix multiplication, and matrix cross product kernels,
  the Knights Landing nodes achieved strong scaling speedups of up to $45$ times
  their single-threaded performance for large matrices.
Furthermore, the best run times achieved on Knights Landing nodes were up to
  $5.25$ times faster than the best run times achieved on Sandy Bridge nodes for
  the same kernels and the largest matrix dimensions.
The best run times achieved by the QR and singular value decompositions were up
  to $6.5$ times faster on Knights Landing than Sandy Bridge for the largest
  tested matrix dimensions, and the Knights Landing nodes achieved strong
  scaling speedups of up to $19$.
The Knights Landing nodes achieved modest strong scaling speedups of up to $3.5$
  for large matrix dimensions when we tested them with the eigendecomposition
  and the linear least squares fit kernels.
Due to the limited scalability of these kernels, the best run times on the
  Knights Landing nodes were only about $30$-$50\%$ smaller than the best run
  times on the Sandy Bridge nodes.
Scalability of the kernels never improved beyond $68$ threads on the Knights
  Landing nodes regardless of matrix dimension.
Tests comparing the performance of a few R kernels and equivalent kernels
  written in C showed that there was very little overhead introduced by the
  interpreter.
Given the low strong scaling achieved by the matrix-vector kernel, alternatives
  to the \texttt{dgemm} function, which R version 3.2.1 uses to implement the
  matrix-vector operation, should be explored.
Microbenchmarks of summary statistics kernels such as mean, variance, and
  covariance computation need to be added to the benchmark to improve its
  coverage of intrinsic R functionality; optimization efforts may need to be
  focused on those kernels depending on performance.
In addition to computational kernels, we also conducted performance tests of
  clustering and neural network training functions and determined that those
  functions had not been parallelized or engineered with parallelized kernel
  functions.
The Sandy Bridge nodes were substantially faster at neural network training with  the \texttt{nnet} function for small numbers of features.
The Sandy Bridge nodes were faster at computing clusters with the \texttt{pam}
  function for $35000$ training vectors or less, but Knights Landing nodes were
  faster beyond $40000$ training vectors.
Performance testing of the neural net training and clustering functions should
  be conducted with larger numbers of features to determine if the performance
  of the Knights Landing nodes will improve with more features.
Furthermore, additional compute intensive machine learning or data analytics
  functionality commonly used in R should eventually be included in the
  benchmark to provide more extensive performance assessments.
Given the lack of parallelism and lower performance of the \textit{nnet} and
  \textit{cluster} packages on the Knights Landing nodes, future optimization
  and parallelization efforts should focus primarily on improving the
  performance of compute intensive packages.
If the performance of the \textit{nnet} and \textit{cluster} packages are an
  indication of the performance of R packages in general, most packages will
  likely need to be vectorized and either parallelized or restructured to rely
  as much as possible on parallelized computational kernels.


\section{Future Work}
% TODO: Move future work content from Discussion section into this section

%\begin{acks}
%  The authors would like to thank...
%  See also \grantsponsor and \grantnum commands in acmguide.pdf
%
%\end{acks}


\bibliographystyle{ACM-Reference-Format}
\bibliography{PEARCKNL}

\end{document}
